\section{Astrophysical Models}

To evaluate possible astrophysical explanations for the two \spock
events, let us first summarize the information we have:

\begin{enumerate}
\item{\label{obs:twoevents} Two events appeared in separate images of a single galaxy,
  separated by $\sim$120 days in the rest frame (240 days in the
  observer frame). \label{itm:TwoEvents}}
\item{\label{obs:timedelay} Lensing models predict a time delay between the two host images
  of $\pm50$ days or less in the observer frame.}
\item{\label{obs:timescale} Each event lasted for $<$15 rest-frame days\label{itm:FastLC}}
\item{\label{obs:luminosity} After correcting for lensing magnification of $\mu\sim30$, both
  events reach a peak luminosity of $\sim10^{41}$ ergs s$^{-1}$
  \textcolor{red}{(NEED TO REFINE THIS NUMBER)}}
\item{\label{obs:xray} \textcolor{red}{ADD IN X-RAY AND GAMMA RAY CONSTRAINTS}}
\end{enumerate}


To explain the observation of two transient episodes in different host
galaxy images (observation \ref{obs:twoevents}), there are four possible
scenarios:

\begin{enumerate}[(A)]
\item{\label{case:microlensing} An intrinsically static source that
  appears transient due to time-varying magnification (e.g. a
  microlensing event).}
\item{\label{case:timedelay}A single transient source (e.g., a stellar
  explosion) that appeared to us twice with an intervening time delay
  due to strong gravitational lensing.}
\item{\label{case:twoexplosions} Two separate events from distinct
  transient sources (e.g., two explosions from unrelated stars).}
\item{\label{case:recurrent} Two separate events from a single
  astrophysical source (e.g., a recurrent nova).}
\end{enumerate}



\noindent In the following subsections we will examine specific
astrophysical sources that might accommodate all of these
observations. Before that, however, we can first exclude some broad
categories of common transients based on the timescale and peak
luminosity.

The very rapid rise and fall of both light curves (observation
\ref{obs:timescale}) is incompatible with any of the normal SN
classes.  For both thermonuclear white dwarf explosions (Type Ia) and
massive star core collapse explosions (Type Ib, Ic, and II) the
optical light curve after reaching peak brightness is primarily
powered by the decay of radioactive \NiFiftySix to \CoFiftySix, which
leads to a minimum decline rate of $\sim$0.1 mag day$^{-1}$.  For Type
II SNe this decay time can be extended into a plateau phase by the
recession of the photosphere via a recombination wave propagint inward
through the ionized H of the expanding outer shell.  In no case can a
normal SN powered by the \NiFiftySix decay chain exhibit the decline
rate of \TODO{measure the decline rate precisely} that has been
observed for \spock.

The observed luminosity (observation \ref{obs:luminosity}) is far too
high for even the most energetic stellar flares, which release a total
energy in the range of $10^{33}$ to $10^{38}$ erg over a span of
minutes to hours \citep{Balona:2012,Karoff:2016}.  Conversely, the
peak luminosity is too low for most Type I or Type II SN, which peak
at $\sim10^{42}$ to $10^{43}$ erg s$^{-1}$.  It is, however, marginally
compatible with several categories of low luminosity transients, such
as luminous red novae, Calcium-rich transients, and the emerging
category of fast optical transients.

\todo{Add references. And expand the
  luminosity discussion slightly}

\todo{Discuss kilonova, .Ia and Fast optical transients}

\todo{Make a figure showing the light curves compared to those fast
  transient models}





\subsection{Microlensing}

In the presence of strong gravitational lensing it is possible to
generate a transient event from lensing effects alone
(Scenario~\ref{case:microlensing}).  In this case the background
source has a steady luminosity but the relative motion of the source,
lens, and observer causes the magnification of that source to change
rapidly with time.

A commonly observed example is the microlensing of a background quasar
by a galaxy-scale lens \citep{Wambsganss:2001, Kochanek:2004}.  In
this optically thick microlensing regime, the lensing potential along
the line of sight to the quasar is composed of many stellar-mass
objects.  Each compact object along the line of sight generates a
separate critical lensing curve, resulting in a complex web of
overlapping critical curves. As all of these lensing stars are in
motion relative to the background source, the web of caustics will
shift across the source position, leading to a stochastic variability
on timescales of months to years.  This scenario is inconsistent with
the observed data, as the two \spock events were far too short in
duration and did not exhibit the repeated ``flickering'' variation
that would be expected from optically thick microlensing.

A second possibility is through an isolated strong lensing event with
a rapid timescale, such as a background star crossing over a lensing
critical curve.  This corresponds to the optically thin microlensing
regime, and is similar to the ``local'' microlensing light curves
observed when stars within our galaxy or neighboring dwarf galaxies
pass behind a massive compact halo object \citep{Paczynski:1986,
  Alcock:1993, Aubourg:1993, Udalski:1993}.  In the case of a star
crossing the caustic of a smooth lensing potential, the amplification
of the source flux would rise (fall) with a characteristic $t^{-1/2}$
profile, and would exhibit a very sharp decline (increase) on the
other side of the caustic \citep{Schneider:1986,MiraldaEscude:1991}.
With a more complex lens comprising many compact objects, the light
curve would exhibit a superposition of many such sharp peaks
\citep{Lewis:1993}.

To generate an isolated microlensing event, the background source
would have to be the dominant source of luminosity in its environment,
meaning it must be a very bright O or B star with mass of order 10
\Msun.  Depending on its age, the size of such a star would range from
a few to a few dozen times the size of the sun.  The net relative
transverse velocity would be on the order of a few 100 km/s, which is
comparable to the orbital velocity of stars within a galaxy or
galaxies within a cluster.  \citet{MiraldaEscude:1991} showed
that---in the case of a smooth cluster potential---the
%timescale
%$\tau$ for the light curve of such a caustic crossing event is
%dictated by the radius of the source, $R$, and the net transverse
%velocity, $v$, of the source across the caustic, as:
%
%\begin{equation}
%  \tau = 6\frac{R}{5\,\Rsun}\frac{300 {\rm km~ s}^{-1}}{v}~\rm{hr}
%\label{eqn:caustic_crossing_time}
%\end{equation}
%
%
%\noindent Thus, the
characteristic timescale of such an event would be on the order of
hours or days, which is in the vicinity of the timescales observed for
the \spock events.  However, this scenario could not plausibly
generate two events with similar decay timescales at distinct
locations on the sky.  This is because a caustic-crossing transient
event must necessarily appear at the location of the lensing critical
curve, but in this case the critical curve most likely passes between
the two \spock locations. At best, a caustic crossing could account
for only one of the \spock events, not both.


\subsection{Single Explosion, Time Delayed}

Any single explosion model (Scenario \ref{case:timedelay}) requires
that the observed transient events must be coincident both in space
and time.  All lensing models evaluated here agree that spatial
coincidence in the source plane is entirely plausible, but coincidence
in time is highly unlikely. For this scenario to be tenable, we would
have to assume that a large systematic bias is similarly affecting all
of the lens models.  While we cannot rule this out, it makes the
single explosion model significantly less tenable.

Under this scenario, the optical observations in January, 2014 and the
infrared observations in August, 2014 are from the same explosion. At
redshift $z=1$ these translate to rest-frame ultraviolet (UV) and
optical wavelengths, respectively, and in Figure \textcolor{red}{TBD}
we examine whether the observed colors in these bands are consistent
with the spectral energy distributions observed for known transients.
\todo{Line up the events at peak, measure colors, compare to SNe,
  novae, kilonovae, .Ia, and fast optical transients}.  We can not
compare the UV to optical colors, since that would require a
correction for differential magnification, and for this scenario we
have already assumed that the lens models are unreliable. \todo{Are
  the colors viable?}



\subsection{Two Separate Sources}

Invoking two separate and unrelated explosions (Scenario
\ref{case:twoexplosions}) would imply that the two events are {\it
  not} coincident at the source plane in either space or time.  This
is consistent with the lens models, which makes this scenario
initially more attractive.  However, removing the spatial coincidence
is problematic.  All five lens models indicate that the locations of
the two events are within \TODO{How many arcsec?  how many pc?}.

\TODO{Make a figure that plots the SED of this host galaxy at the two
  spock locations and at the center of the host galaxy image 11.3.
  Quantitatively assess the probability that the three SEDs are
  consistent with being at the same location on the source plane. }

Having two unrelated explosions occurring in the same
year within such a small physical area would be plausible if the
explosions were from a very common source, but the rapid light curve
already rules out all of the common categories of supernova explosions.
Since we have not observed any similar fast transients in any other
galaxy throughout the Frontier Fields survey, to accept this scenario
we would have to conclude that the \spock host galaxy is a very
unusual physical environment that provided uniquely fertile ground for
such transients.

\TODO{Evaluate the star formation and metallicity of this host galaxy
  to determine whether the host is very unique.}

Nevertheless, let us accept the premise that this host has generated
two separate rare explosions in the same year, and consider what rare
explosion categories could be separately consistent with the observed
light curves.

\TODO{Evaluate kilonova, .Ia and Fast Optical transient light curves
  compared to the Spock light curves}

\TODO{Evaluate the likelihood of two rare explosions appearing in the
  same lensed host galaxy, comparing to rates from PS1, etc.}


\subsection{Two Events from the Same Source}

The final alternative is to allow two separate explosive events
powered by the same astrophysical source (Scenario
\ref{case:recurrent}).  In this case the observed events must be
spatially coincident but not coincident in time, which is fully
consistent with all lens model constraints.  Any transient event that
appears at the \spockone\ location must also appear at the
\spocktwo\ location, separated in time by a gravitational lensing time
delay.  Our lens models suggest that the time delay would be on the
order of 10--50 days, so this scenario supposes that those
``gravitational echoes'' were simply not observed, as they landed in
one of the long periods without \HST observations on this field.

Adopting this hypothesis immediately rules out any catastrophic
explosive events---such as a supernova or neutron star merger---in
which the progenitor system is completely destroyed or disrupted. This
scenario also does not admit any of the category of {\it variable}
sources (e.g. Cepheids, RR Lyrae, or Mira variables) that exhibit
periodic changes in flux due to pulsations of the stellar photosphere
but do not have sharp, isolated transient episodes.

There are two broad categories of astrophysical sources with
recurrent explosive events that might fit this scenario.  The first is
active galactic nuclei (AGN), in which transient episodes can be
driven by clumps of matter falling onto the accretion disk of a
supermassive black hole.  The second is a recurrent stellar explosion,
such as a recurrent nova (RN) or luminous blue variable (LBV) star.
In a RN system a white dwarf star accretes matter from a close binary
companion and experiences a surface explosion that leaves the system
intact to restart the cycle.  The LBV stars are very massive evolved
stars ($\sim$10-100 \Msun) that exhibit occasional outbursts or
eruptions associated with significant mass loss episodes -- although
the exact physical mechanism for these events remains unclear.

The AGN hypothesis is disfavored principally due to the quiescence of
the \spock sources between episodes. In addition to stochastic and
brief transient events, most AGN typically also exhibit slower
variation on much longer timescales, which is not observed at either
of the \spock locations. Furthermore, the spectrum of the \spock host
galaxy shows none of the broad emission lines that are often (though
not always) observed in AGN.  Finally, an AGN would necessarily be
located at the center of the host galaxy. The severe distortion of the
host galaxy images makes it impossible to identify the location of the
host center from the galaxy morphology, and any spatial reconstruction
from the lens models is not precise enough for a useful test.
However, since gravitational lensing is achromatic, if the \spock
positions are coincident with the host galaxy center, then the color
of the galaxy at each \spock location in images 11.1 and 11.2 should
be consistent with the color at the center of the less distorted image
11.3.  Figure~\ref{fig:HostGalaxyColor} shows that the rest-frame U-V
color for both \spockone and \spocktwo is in the range
\textcolor{red}{CHECK THIS:} $0.65\pm0.15$, while the center of the
galaxy in image 11.3 has $U-B=0.35\pm0.15$. Although by no means
definitive, this suggests that the \spock events were not located at
the physical center of the host galaxy.

\todo{Discuss recurrent novae}

\todo{Discuss LBV eruptions}



