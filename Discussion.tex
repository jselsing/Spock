


If we accept for the moment that the source of the \spock transient
events is a physically extreme nova system, then we find a significant
problem with the nova 

The most luminous novae observed in the local
universe were all somewhat slower in their decline rate.

SN 2010U has t2 = 3.5 ± 0.3 days and the rise time is unconstrained.

t2 = 6 ± 1 days (L91; Schwarz et al. 2001) and t2 = 9.5 days (M31N; Shafter et al. 2009). By comparison, 

L91 (Della Valle 1991; Schwarz et al. 2001; Williams et al. 1994) and M31N-2007-11d (Shafter et al. 2009)

The preferred peak luminosity of
$10^{41}$ erg s$^{-1}$ that we have inferred for \spock would imply
that this is among the most luminous novae ever observed.



extremely luminous novae \citep{Czekala:2013}
was an Fe II nova, inconsistent with the usual picture of He/N novae as the brightest with the most massive WDs.

Two other luminous Fe ii type novae have been studied extensively: , hereafter M31N.

The rise to maximum of L91 is among the longest for novae on record, with a peak of Mv = −10.0 mag. The light curve of L91 shown here is drawn from the photometry published in the circulars (Shore et al. 1991; Gilmore 1991; Gilmore et al. 1991; Liller et al. 1991; Della Valle et al. 1991). Shafter et al. (2009) set a lower limit of four days on the rise time for M31N from quiescence to a maximum light of MV ≃ −9.5 mag.

Both novae declined rapidly from maximum light with 



Furthermore, the spectroscopic classification
of Novae is also correlated with their luminosity and light curve
decline time: those showing prominent He/N features are brighter and
fade faster than those with spectra dominated by Fe II lines.


Shafter et al 2011:

``more luminous novae generally fade the fastest and [...]  He/N novae
are typically faster and brighter than their Fe II counterparts. In
addition, we find a weak dependence of nova speed class on position in
M31, with the spatial distribution of the fastest novae being slightly
more extended than that of slower novae.''


Recurrent novae make up roughly 25\% of the nova population,
masquerading as CNe \citep{Pagnotta:2014}.
