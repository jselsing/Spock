\section{Discussion}
\label{sec:Discussion}

We have found that the \spock transient events are phenomenologically
most consistent with RNe and LBVs.  Let us now consider
the astrophysical implications of these two possible classifications.

\subsection{Physical Implications of the RN Model}

The physical limits of the RN model are best evaluated by combining
the two key observables of recurrence period and peak brightness. In
this examination we rely on a pair of papers that evaluated an
extensive grid of nova models through multiple cycles of outburst and
quiescence \citep{Prialnik:1995,Yaron:2005}.
Figure~\ref{fig:RecurrentNovaRecurrenceComparison} plots first the RN
outburst amplitude (the apparent magnitude between outbursts minus the
apparent magnitude at peak) and then the peak luminosity against the
log of the recurrence period in years.
% The observations for \spock 
% are shown in comparison to observed RNe (crosses) and theoretical
% models (circles) from \citet{Yaron:2005}.
For the \spock events we can only measure a lower limit on the
outburst amplitude, since the presumed progenitor star is unresolved,
so no measurement is available at
quiescence. Figure~\ref{fig:RecurrentNovaRecurrenceComparison} shows
that a recurrence period as fast as one year is expected only for a RN
system in which the primary WD is both very close to the Chandrasekhar
mass limit (1.4 \Msun) and also has an extraordinarily rapid mass
transfer rate ($\sim10^{-6}$ \Msun yr$^{-1}$).  The models of
\citet{Yaron:2005} suggest that such systems should have a very low
peak amplitude (barely consistent with the lower limit for \spock) and
a low peak luminosity ($\sim$100 times less luminous than the \spock
events).

The closest analog for the \spock events from the population of known
RN systems is the nova M31N\,2008a-12.  \citet{Kato:2015} provided a
theoretical model that can account for the key observational
characteristics of this remarkable nova: the very rapid recurrence
timescale ($<$1 yr), fast optical light curve ($\t2\sim2$ days), and
short supersoft x-ray phase \citep[6-18 days after optical
  outburst][]{Henze:2015a}.  To match these observations,
\citeauthor{Kato:2015} invoke a 1.38 \Msun white dwarf primary,
drawing mass from a companion at a rate of $1.6\times10^{-7}$ \Msun
yr$^{-1}$.  This is largely consistent with the theoretical
expectations derived by \citet{Yaron:2005}, and reinforces the
conclusion that a combination of a high mass white dwarf and efficient
mass transfer are the key ingredients for rapid recurrence and short
light curves. The one feature that can not be effectively explained
with this scenario is the peculiarly high luminosity of the \spock
events -- even after accounting for the very large uncertainties.  If
the \spock transients are caused by a single RN system, then that
progenitor system would be the most extreme WD binary yet known.

\subsection{Physical Implications of the LBV Model}

For the LBV scenario, the observed \spock events would also stand out
as quite extreme.  The observed rise and decline times for \spock
would place both among the most rapid LBV major eruptions ever seen.
The peak luminosities of both \spockone and \spocktwo are similar to
the observed luminosities of rapid, bright outbursts seen in LBVs such
as SN 2009ip and NGC3432-LBV1. However, the upper edge of the range of
plausible peak luminosities for both \spock events reaches $10^{42}$
erg s$^{-1}$, which would be an order of magnitude more luminous than
any rapid outburst from those two nearby LBVs.

The precise physical mechanism for LBV outbursts is still not fully
understood.  LBV stars such as \etacar show clear evidence of ejected
shells of gas, and very massive stars are known to undergo extensive
mass loss as they evolve toward eventual explosion as a SN.  This has
led to the canonical model of LBV transient events as being the
optical signature of an eruptive mass loss episode.  Such mass loss
could arise from a variety of direct mechanisms, such as
continuum-driven super-Eddingtion winds \citep{Smith:2006},
pulsational pair instability ejections \citep{Woosley:2007}, and shock
heating of stellar envelopes from internal shell-burning instabilities
\citep{Dessart:2010}.  This is far from an exhaustive list, and none
of these explanations are entirely sufficient to account for all of
the observed diversity of LBV behaviors or the structural complexity
the most well-studied LBVs \citep[e.g.][]{Smith:2011b, Kochanek:2012}.

Although we do not have a complete physical model in hand, we can
nevertheless explore some of the physical implications of an LBV
classification for the two \spock events.  We first make a rough
estimate of the total radiated energy, which can be computed using the
decline timescale $t_2$ and the peak luminosity $L_{\rm pk}$ following
\citet{Smith:2011b}:

\begin{equation}
  \label{eqn:Erad}
  E_{\rm rad} = \xi \t2 \Lpk,
\end{equation}

\noindent where $\xi$ is a factor of order unity that depends on the
precise shape of the light curve.\footnote{Note that
  \citet{Smith:2011b} used $t_{1.5}$ instead of $t_2$, which is
  amounts to a different light curve shape term, $\xi$.}  Adopting
\Lpk$\sim10^{41}$ erg s$^{-1}$ and \t2$\sim$2 days (as shown in
Figure~\ref{fig:PeakLuminosityDeclineTime}), we find that the total
radiated energy isi $E_{\rm rad}\sim10^{46}$ erg.  A realistic range
for this estimate would span $10^{44}<E_{\rm rad}<10^{47}$ erg, due to
uncertainties in the magnification, bolometric luminosity correction,
decline time, and light curve shape (in roughly that order of
importance). These uncertainties notwithstanding, our crude estimate
does fall well within the range of plausible values for the total
radiated energy of a major LBV outburst.

If LBV eruptions are driven by significant mass ejection events, then
the energy budget would also include a substantial amount of kinetic
energy imparted to the ejected gas shell. Without spectroscopic
information from the \spock transients we can not place any realistic
estimate on the kinetic energy. Nevertheless, we can take the radiated
energy as a rough lower limit on the total energy release and ask what
timescale would be required for a massive star to build up that amount
of energy. This approach assumes that the energy released in an LBV
eruption is generated slowly in the stellar interior and is in some
way ``bottled up'' by the stellar envelope over months or years,
before being released in a rapid mass ejection.  The ``build-up''
timescale to match the radiative energy release is then

\begin{equation}
  \label{eqn:trad}
t_{\rm rad} = \frac{E_{\rm rad}}{L_{\rm qui}} = \t2 \frac{\xi\Lpk}{L_{\rm qui}},
\end{equation}

\noindent where $L_{\rm qui}$ is the luminosity of the LBV progenitor
star during quiescence. For the \spock events we have no useful
constraint on the quiescent luminosity, but for evaluating the LBV
scenario we can assume it is similar to the local LBVs whose
progenitors have been directly observed.  This gives a range for the
radiative build-up timescale between $t_{\rm rad}\sim30$ days if the
progenitor is \etacar-like ($M_V\sim-12$), or $t_{\rm rad}\sim20$
years if it is similar to the faintest known LBV progenitors (e.g. SN
2010dn, with $M_V\sim-6$).  

A more informative alternative approach is to assert that the build-up
timescale for \spock corresponds to the observed rest-frame lag
between the two events, roughly 120 days. Adopting $\Lpk=10^{41}$ erg
s$^{-1}$ and $\t2=2$ days, if we assume $t_{\rm rad}=120$ days we can
infer that the quiescent luminosity of the \spock progenitor would be
$L_{\rm qui}\sim10^{39.5}$ erg s${-1}$ ($M_V\sim-10$).  This is a very
reasonable quiescent luminosity value for the massive ($M>10\Msun$)
progenitor stars expected for LBVs.

Although the above discussion shows that the observations of the
\spock transients are largely consistent with the observed
characteristics of known LBV systems, this does not mean that we have
a viable physical model to explain these events. Rapid transient
episodes in LBVs such as SN 2002kg and SN 2009ip may best be explained
by a sudden ejection of an optically thick shell
\citep[e.g.]{Smith:2010,Smith:2011b}, or by some form of S Dor-type
variability \citep{Weis:2005,VanDyk:2006,Foley:2011}, which may be
driven by stellar pulsation rather than mass ejection
\citep{VanGenderen:1997,VanGenderen:2001}.

For massive stars such as \etacar at its great eruption and the
rapidly varying SN 2009ip, the effective photospheric radius during
eruption must have been comparable to the orbit of Saturn
\citep[$10^{14}$ cm][]{Davidson:1997,Smith:2011,Foley:2011}.  With
observed photospheric velocities of order 500 km s$^{-1}$ for such
events, the dynamical timescale is on the order of tens to hundreds of
days.  The very rapid light curves of both \spock events will add to
the already challenging task of developing a coherent theoretical
explanation for the physical mechanisms that drive the great eruptions
and the S Dor-type variation of LBVs.

%To examine the temperature and total energy output, we first make a
%set of (admittedly unfounded) assumptions: (1) the two outbursts had a
%very similar SED; (2) the last observed epoch for each event
%corresponds to the same phase relative to the true epoch of peak
%brightness; and (3) the lensing magnifications for the two events are
%the same.  These simplifying assumptions allow us to jointly apply the
%optical observations of \spockone and the NIR observations of
%\spocktwo as constraints on the SED in any given epoch.  We then set
%an assumption for the epoch of peak brightness, make another
%assumption for the magnification of both events, and then fit a
%blackbody to the resulting extrapolated SED. From this blackbody fit
%we derive a temperature and integrate to get an estimate of the
%pseudo-bolometric luminosity.  The resulting inferred physical
%parameters are plotted in Figure~\ref{fig:DerivedPhysicalParameters}.
