Peak luminosity vs. decline time for \spock and other astrophysical
transients.  Constraints for \spockone are plotted as overlapping cyan
and blue bands, corresponding to independent constraints drawn from
the F435W and F814W light curves, respectively.  For \spocktwo the
scarlet and maroon bands show constraints from the F125W and F160W
light curves, respectively.  Each band encompass the range of possible
peak luminosities and time to decline by 3 magnitudes. The width and
height of these bands incorporates the uncertainty due to
magnification (we adopt $10<\mu<100$) and the time of peak (using
linear fits as shown in Figure~\ref{fig:LinearLightCurveFits}).
Ellipses and rectangles mark the luminosity and decline-time regions
occupied by other transient classes.  Filled shapes show the empirical
bounds for transients with a substantial sample of known events. Open
ellipses mark theoretical expectations for rare transients that lack a
significant sample size: the ``.Ia'' class of white dwarf He shell
detonations \citep{Bildsten:2007,Shen:2010} and the kilonova class
from neutron star
mergers \citep{Kulkarni:2005,Tanvir:2013,Kasen:2015}.
\label{fig:PeakLuminosityDeclineTimeWide}
