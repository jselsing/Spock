Peak luminosity vs. decline time for \spock and other rapidly evolving transients.  Constraints for \spockone are plotted as overlapping cyan and blue bands, corresponding to independent constraints drawn from the F435W and F814W light curves, respectively.  For \spocktwo the scarlet and maroon bands show constraints from the F125W and F160W light curves, respectively.  Each band encompass the range of possible peak luminosities and time to decline by 3 magnitudes. The width and height of these bands incorporates the uncertainty due to magnification (we adopt $10<\mu<100$) and the time of peak (using linear fits as shown in Figure~\ref{fig:LinearLightCurveFits}).   Other examples of rapidly evolving transients are shown for comparison, with the "optical fast transients" from the Pan-STARRS1 survey appearing as circles in the upper right \citep{Drout:2014a}, and various novae in the lower portion of the figure. Classical novae appear as cyan squares \citep{Downes:2000}, galactic recurrent novae as blue and cyan diamonds \citep{Schaefer:2010}, and the rapid recurrence nova M31N200812a as large cyan and orange diamonds.  For all of these points the marker color indicates the band-pass of the light curve used to derive the luminosity and decline time: blue indicates B band, cyan is V band, green is g band, and orange corresponds to r or R band.