\section{Classification}
\label{sec:Classification}

For transients that do not easily fall within familiar categories, a
useful starting point for classification is to examine the object in
the phase space of peak luminosity versus decline time \citep[see,
  e.g.,][]{Kasliwal:2010}.  To infer the luminosity and decline time
for each \spock event, we combine the linear fits to the light curves
(shown in Figure~\ref{fig:LinearLightCurveFits}) with the predicted
range of lensing magnifications
(Figure~\ref{fig:LensModelContours}. For any assumed value for the
time of peak brightness, the light curve fits give us an estimate of
the ``observed'' peak magnitude and a corresponding rise-time and
decline-time measurement.  We then convert this extrapolated peak
magnitude to a luminosity (e.g., $\nu L_\nu$ in erg s$^{-1}$) by
first correcting for the luminosity distance assuming a standard \LCDM
cosmology, and then accounting for an assumed lensing magnification,
$\mu$.  At the end of all this, we have a grid of possible peak
luminosities for each event as a function of magnification and time of
peak (or, equivalently, the decline time).

Figure~\ref{fig:PeakLuminosityDeclineTimeWide} shows the resulting
constraints on the peak luminosity and the decline time, which we
quantify as $t_2$, the time over which the transient declines by 2
magnitudes.  Shaded green and red bands represent the \spockone and
\spocktwo events, respectively, and in each case they incorporate the
allowed range for time of peak (see
Figure~\ref{fig:LinearLightCurveFits}) and the lensing magnification
($10<\mu<100$) as reported in Table~\ref{tab:LensModelPredictions}.
The two events are largely consistent with each other, and if both
events are representative of a single system (or a homogeneous class)
then the most likely peak luminosity and decline time (the region with
the most overlap) would be $L_{\rm pk}\sim10^{41}$ ergs/s and
$t_3\sim1.8$ days.

\subsection{Comparison to SN and SN-like Explosion Models}

In Figure~\ref{fig:PeakLuminosityDeclineTimeWide} we also demarcate
regions of the luminosity - decline time phase space occupied by known
or theorized transients.
The colored regions along the right
side of Figure~\ref{fig:PeakLuminosityDeclineTimeWide} mark the
luminosity and decline times for SNe and SN-like transients, showing
that the very rapid rise and fall of both \spock light curves is
incompatible with any of the normal SN classes.  For both
thermonuclear white dwarf explosions (Type Ia) and the core-collapse
explosions of massive stars (Type Ib, Ic, and II) the optical light
curve after reaching peak brightness is primarily powered by the decay
of radioactive \NiFiftySix to \CoFiftySix, which leads to a minimum
decline rate of $\sim$0.1 mag day$^{-1}$.  For Type II SNe this decay
time can be extended into a plateau phase by the recession of the
photosphere via a recombination wave propagating inward through the
ionized H of the expanding outer shell.  In no case can a normal SN
powered by the \NiFiftySix decay chain exhibit the decline rate of
$t_3<8$ days that has been observed for \spock.  Furthermore, the
observed peak luminosities for both \spock events are too low for most
Type I or Type II SN, which peak at $\sim10^{42}$ to $10^{43}$ erg
s$^{-1}$.

Figure~\ref{fig:PeakLuminosityDeclineTimeWide} also shows that the
\spock events are far too faint and far too fast to belong to the less
well-understood classes of Superluminous SNe
\citep{Gal-Yam:2012,Arcavi:2016}, which have peak luminosities
$>10^{43}$ erg s$^{-1}$ and take at least 10's of days to decline by 2
magnitudes.  \spockone and \spocktwo also decline too rapidly to match
the most common categories of peculiar SNe and ``SN-like'' transients
such as Type Iax SNe \citep{Foley:2013a}, fast optical transients
\citep{Drout:2014}, Ca-rich SNe
\citep{Filippenko:2003,Perets:2011,Kasliwal:2012}, or Luminous Red
Novae \citep[also called intermediate luminosity red
  transients][]{Munari:2002,Kulkarni:2007,Kasliwal:2011}.

Dashed boxes in Figure~\ref{fig:PeakLuminosityDeclineTimeWide}
represent categories of stellar explosions that have been
theoretically predicted and extensively modeled, but for which very
few viable candidates have actually been observed.  One of these is
the ``.Ia'' class, due to explosions of a He shell on the surface of a
white dwarf \citep{Bildsten:2007}.  Theoretical .Ia models suggest
that after an initial short peak (3-5 days) driven by the rapid
radioactive decay of \ion{Cr}{48} and \ion{Fe}{52} at the exterior of
the ejecta, a secondary decline phase kicks in, powered by the slower
\ion{Ni}{56} decay chain \citep{Shen:2010}.  There are only two .Ia
candidates in the literature \citep{Poznanski:2010,Kasliwal:2010}, so
we do not have enough objects to empirically constrain the range of
.Ia light curve shapes.  The best we can do is to conclude from the
available models that a .Ia light curve would be brighter and slower
than the observed \spock light curves.

A second dashed box in Figure~\ref{fig:PeakLuminosityDeclineTimeWide}
represents the ``kilonova'' class (also called ``Macronovae'' or
``mini-supernovae''), which are theorized to be generated by the
merger of two neutron stars (NSs). A NS+NS merger can drive a
relativistic jet that may be observed as a Gamma Ray Burst (GRB) and
would emit gravitational waves.  These may also be accompanied by a
very rapid optical light curve (the kilonova component) that is driven
by the radioactive decay of r-process elements in the ejecta
\citep{Li:1998a,Kulkarni:2005}.  To date there are two cases of fast
optical transients associated with GRB events, which have been
interpreted as possible kilonovae \citep{Perley:2009,Tanvir:2013}.
The \spock transients fall within the range of theoretically predicted
peak luminosity and decline times for kilonovae. However, the rise
time for the \spockone event is at least 5 days in the rest-frame,
which is significantly longer than the $<1$ day rise expected for a
kilonova \citep[e.g.][]{Metzger:2010,Barnes:2013,Kasen:2015}.
Furthermore, both \spock events are either significantly fainter or
faster than the optical light curves for the two existing kilonova
candidates.

An additional challenge to all of the models discussed above, from
supernova to .Ia to kilonova, is that all of these stellar explosions
are one-time events.  For the common categories of Type Ia, Type Iax,
Core Collapse, and Superluminous SNe, the progenitor stars are
completely destroyed by the explosion.  Some SN-like explosions --
such as the He shell explosions in the .Ia SN model -- could leave the
progenitor star at least partially intact.  However, even then the
mass lost from the system would be sufficient to fatally disrupt the
mass accretion process, preventing the system from evolving back to
generate another similar explosion.  These catastrophic stellar
explosions are also intrinsically rare (fewer than 1 event per century
per 10$^11$ \Msun), as they require progenitor systems that start as high
mass stars and/or close binary systems that can sustain mass transfer.
Thus, the only way to reconcile a cataclysmic explosion model with the
two observed \spock events is to either (a) invoke a highly
serendipitous occurrence of two unrelated explosions in the same host
galaxy in the same year, or (b) assert that the two events are two
images of the same explosive event, appearing to us separately only
because of a gravitational lensing time delay \citep[as was the case
  for the 5 images of SN Refsdal][]{Kelly:2015a,Kelly:2016}.

For the former scenario of two unrelated explosions, we have already
ruled out all of the most common categories of stellar explosions
based on the properties of the light curve.  The only models that can
accommodate the observed rapid light curves are the kilonova model and
perhaps the .Ia scenario.  As there are no measured rates of .Ia or
kilonovae, it is difficult to accurately quantify the likelihood of
detecting such rare events. However, we have observed no other
transients with similar luminosity and light curve shapes in the
high-cadence surveys of 5 other Frontier Fields clusters.  Thus, it
appears unreasonable to allow that two such explosions would occur in
the same galaxy in a single year.

Scenario (b), in which a lensing time delay causes the two events, is
also highly implausible.  We have seen in
Section~\ref{sec:LensingModels} that none of the \macs0416 lens models
predict an 8 month time delay between appearances in image 11.1 and
11.2.  This is represented in Figure~\ref{fig:SpockDelayPredictions},
where we have plotted the light curves for the two transient events,
along with shaded vertical bars marking the time delay predictions of
all models.
%The lens models are broadly consistent with each other, predicting
%that the lensing time delay between images 11.1 and 11.2 is on the
%order of $\pm$60 days, far short of the 238 day lag that was observed
%between \spockone\ and \spocktwo.
To accept the alternative single-explosion explanation for \spock, we
would have to assume that a large systematic bias is similarly
affecting all of the lens models.  While we cannot rule out such a
bias, the consistency of the lens modeling makes any catastrophic
(i.e., non-repeating) explosion model less tenable.



\subsection{Recurrent Nova Model}

Novae are represented in
Figure~\ref{fig:PeakLuminosityDeclineTimeWide} as a grey band, which
traces the maximum magnitude - rate of decline (MMRD) relation.  Nova
explosions occur in binary star systems in which the more massive star
is a white dwarf that accretes matter from its companion, which may be
a main sequence dwarf or evolved giant star overfilling its Roche
Lobe. The white dwarf builds up a dense layer of H-rich material on
its surface until the high pressure and temperature triggers nuclear
fusion, resulting in a surface explosion that causes the white dwarf
to brighten by several orders of magnitude, but does not completely
disrupt the star. In a recurrent nova (RN) system, the mass transfer
from the companion to the white dwarf restarts after the explosion, so
the cycle may begin again and repeat after a period of months or
years.

The seminal work of \citet{Zwicky:1936} and \citet{McLaughlin:1939}
first showed that more luminous novae within the Milky Way tend to
have more rapidly declining light curves, which is now the basis of
the maximum-magnitude versus rate-of-decline (MMRD) relationship. The
basic form of the MMRD relation has been theoretically attributed to a
dependence of the peak luminosity on the mass of the accreting white
dwarf \citep[e.g.][]{Livio:1992}.  Studies of extragalactic novae
reaching as far as the Virgo cluster have shown that the MMRD relation
is broadly applicable to all nova populations, though with significant
scatter
\citep[e.g.][]{Ciardullo:1990,DellaValle:1995,Ferrarese:2003,Shafter:2011}.
Amidst that scatter, there may also be sub-populations of novae that
deviate from the traditional MMRD form \citep{Kasliwal:2011}, and
recurrent novae (RNe) in particular may be poorly represented by the
MMRD \citep{Shafter:2011,Hachisu:2015}

In Figure~\ref{fig:PeakLuminosityDeclineTimeWide}, the dark grey
region follows the empirical constraints on the MMRD from
\citet{DellaValle:1995}, and the wider light grey band allows for the
increased scatter about that relation that has been noted from more
extensive surveys of novae in the Milky Way \citep{Downes:2000}, M31
\citep{Shafter:2011} and elsewhere in the local group
\citep{Kasliwal:2011}.  Nova outbursts can exhibit decline times from
$\sim$1 day to many months, so the timescale of the \spock light
curves can easily be accommodated by the nova scenario. However, the
peak luminosities inferred for the \spock events are larger than any
known novae, perhaps by as much as 2 orders of magnitude.

Figure~\ref{fig:PeakLuminosityDeclineTime} shows a narrower slice of
the same phase space as in
Figure~\ref{fig:PeakLuminosityDeclineTimeWide}, zooming in on the
``fast and faint'' region from the lower left corner.  The observed
constraints from the two published kilonova candidates are shown,
which provide only lower limits on the peak luminosity
\citep{Tanvir:2013}, or the decline timescale \citep{Perley:2009}.
The sample of observed nova outbursts (shown as solid points)
demonstrates the observed scatter about the MMRD relation.  

One primary first line of evidence supporting the nova hypothesis
comes from the \spock light curves. Some RN light curves are similar
in shape to the \spock episodes, exhibiting a sharp rise ($<10$ days
in the rest-frame) and a similarly rapid decline.
Figure~\ref{fig:RecurrentNovaLightCurveComparison} compares the \spock
light curves to template light curves from RNe within our galaxy and
in M31.  There are 10 known RNe in the Milky Way galaxy, and 7 of
these exhibit outbursts that decline rapidly, fading by 2 magnitudes
in less than 10 days \citep{Schaefer:2010}
% U Sco, V2487 Oph, V394 CrA, T CrB, RS Oph, V745  Sco, and V3890 Sgr.
The gray shaded region in
Figure~\ref{fig:RecurrentNovaLightCurveComparison} encompasses the V
band light curve templates for all 7 of these events, from
\citet{Schaefer:2010}.  The Andromeda galaxy (M31) also hosts at least
one RN with a rapidly declining light curve.  The 2014 eruption of
this well-studied nova, M31N 2008a-12, is shown as a solid black line
in Figure~\ref{fig:RecurrentNovaLightCurveComparison}, fading by 2
mags in less than 3 days.  This comparison demonstrates that the
sudden disappearance of both of the \spock transient events is fully
consistent with the eruptions of known RNe in the local universe.

The second reason to consider the RN model is that it provides a
natural explanation for having two separate explosions that are
coincident in space but not in time.  If \spock is a RN, then the two
observed episodes can be attributed to two distinct nova eruptions,
and the gravitational lensing time delay does not need to match the
observed 8 month separation between the January and August 2014
appearances.

Although {\it qualitatively} consistent with the 8-month separation,
the RN model is strained by a quantitative assessment of the
recurrence period. If \spock is indeed a RN at $z=1$, then the
recurrence timescale in the rest-frame is $120\pm30$ days ($3-5$
months), where the uncertainty accounts for the $1\sigma$ range of
modeled gravitational lensing time delays.  This would be a singularly
rapid recurrence period, significantly faster than all 11 RNe in our
own galaxy, which have recurrence timescales ranging from 15 years (RS
Oph) to 80 years (T CrB). For the 5 galactic RNe with a rapidly
declining outburst light curve (U Sco, V2487 Oph, V394 CrA, T CrB, and
V745 Sco), the median recurrence timescale is 21 years.  The fastest
measured recurrence timescale belongs to the Andromeda galaxy nova
M31N 2008a-12, which has exhibited a new outburst every year from
2009-2015
\citep{Tang:2014,Darnley:2014,Darnley:2015,Henze:2015,Henze:2015a}. Although
this M31 record-holder demonstrates that very rapid recurrence is
possible, classifying \spock as a RN would still require a very
extreme mass transfer rate to accommodate the $<1$ year recurrence.

A second major concern with the RN hypothesis for \spock is apparent
in Figure~\ref{fig:PeakLuminosityDeclineTime}, which shows that the
two \spock events are substantially brighter than all known novae --
perhaps by as much as 2 orders of magnitude.  One might attempt to
reconcile the \spock luminosity more comfortably with the nova class
by assuming a significant lensing magnification for one of the two
events. This would drive down the intrinsic luminosity, perhaps to
$\sim10^{40}$ erg s${-1}$, on the edge of the nova region.  However,
this assumption implicitly moves the lensing critical curve to be
closer to the \spock event in question.  That pulls the critical curve
away from the other \spock position, which makes that second event
{\rm more inconsistent} with observed nova peak luminosities.  



\subsection{Non-explosive Astrophysical Transients}

There are several categories of astrophysical transients that we have
neglected so far, but which cannot accommodate the observations of the
\spock transients.  We may first dismiss any of the category of {\it
  periodic} sources (e.g. Cepheids, RR Lyrae, or Mira variables) that
exhibit regular changes in flux due to pulsations of the stellar
photosphere. These variable stars do not exhibit sharp, isolated
transient episodes that could match the \spock light curve shapes.

We can also rule out active galactic nuclei (AGN), in which brief
transient episodes (a few days in duration) may be observed from
X-ray to infrared wavelengths.
The AGN hypothesis
for \spock is disfavored for three primary reasons:
%principally due to the quiescence of the
%\spock sources between the two observed episodes.
First, AGN that exhibit short-duration transient events also typically
exhibit slower variation on much longer timescales, which is not
observed at either of the \spock locations. Second, the spectrum
of the \spock host galaxy shows none of the broad emission lines that
are often (though not always) observed in AGN.  Third, an AGN would
necessarily be located at the center of the host galaxy.
%The severe distortion of the
%host galaxy images makes it impossible to identify the location of the
%host center in images 11.1 and 11.2 from the galaxy morphology. Any
%spatial reconstruction at the source plane from the lens models would
%not be not precise enough for a useful test.  However, since
%gravitational lensing is achromatic, if the \spock positions are
%coincident with the host galaxy center, then the color of the galaxy
%at each \spock location in images 11.1 and 11.2 should be consistent
%with the color at the center of the less distorted image 11.3.
In Figure~\ref{fig:HostGalaxyColor} and Section~\ref{sec:HostGalaxy}
we saw that there are minor differences in the host galaxy properties
(i.e. rest-frame U-V color and mean stellar age) from the \spockone
and \spocktwo locations to the center of the host galaxy at image 11.3
Although by no means definitive, this suggests that the \spock events
were not located at the physical center of the host galaxy, and
therefore are not related to an AGN.

Stellar flares provide a third very common source for optical
transient events. Relatively mild stellar flares may be caused by
magnetic activity in the stellar atmosphere, and the brightest flare
events (so-called ``superflares'') may be generated by perturbations
to the stellar atmosphere via interactions from a disk, a binary
companion, or a planet.  In these circumstances the stars release a
{\em total} energy in the range of $10^{33}$ to $10^{38}$ erg over a
span of minutes to hours \citep{Balona:2012,Karoff:2016}. This falls
far short of the observed energy release from the \spock transients,
so we can also dismiss stellar flares as implausible for this source.

\subsection{Microlensing}

In the presence of strong gravitational lensing it is possible to
generate a transient event from lensing effects alone.  In this case
the background source has a steady luminosity but the relative motion
of the source, lens, and observer causes the magnification of that
source to change rapidly with time.

A commonly observed example is the microlensing of a bright background
source (a quasar) by a galaxy-scale lens \citep{Wambsganss:2001,
  Kochanek:2004}.  In this optically thick microlensing regime, the
lensing potential along the line of sight to the quasar is composed of
many stellar-mass objects.  Each compact object along the line of
sight generates a separate critical lensing curve, resulting in a
complex web of overlapping critical curves. As all of these lensing
stars are in motion relative to the background source, the web of
caustics will shift across the source position, leading to a
stochastic variability on timescales of months to years.  This
scenario is inconsistent with the observed data, as the two \spock
events were far too short in duration and did not exhibit the repeated
``flickering'' variation that would be expected from optically thick
microlensing.

A second possibility is through an isolated strong lensing event with
a rapid timescale, such as a background star crossing over a lensing
critical curve.  This corresponds to the optically thin microlensing
regime, and is similar to the ``local'' microlensing light curves
observed when stars within our galaxy or neighboring dwarf galaxies
pass behind a massive compact halo object \citep{Paczynski:1986,
  Alcock:1993, Aubourg:1993, Udalski:1993}.  In the case of a star
crossing the caustic of a smooth lensing potential, the amplification
of the source flux would increase (decrease) with a characteristic
$t^{-1/2}$ profile as it moves toward (away from) the caustic. This
slowly evolving light curve transitions to a very sharp decline (rise)
when the star has moved to the other side of the caustic
\citep{Schneider:1986,MiraldaEscude:1991}.  With a more complex lens
comprising many compact objects, the light curve would exhibit a
superposition of many such sharp peaks \citep{Lewis:1993}.

To generate an isolated microlensing event, the background source
would have to be the dominant source of luminosity in its environment,
meaning it must be a very bright O or B star with mass of order 10
\Msun.  Depending on its age, the size of such a star would range from
a few to a few dozen times the size of the sun.  The net relative
transverse velocity would be on the order of a few 100 km/s, which is
comparable to the orbital velocity of stars within a galaxy or
galaxies within a cluster.  In the case of a smooth cluster potential---the
%timescale
%$\tau$ for the light curve of such a caustic crossing event is
%dictated by the radius of the source, $R$, and the net transverse
%velocity, $v$, of the source across the caustic, as:
%
%\begin{equation}
%  \tau = 6\frac{R}{5\,\Rsun}\frac{300 {\rm km~ s}^{-1}}{v}~\rm{hr}
%\label{eqn:caustic_crossing_time}
%\end{equation}
%
%
%\noindent Thus, the
characteristic timescale of such an event would be on the order of
hours or days \citet{Chang:1979,Chang:1984,MiraldaEscude:1991}, which
is in the vicinity of the timescales observed for the \spock events.
However, if we apply this scenario to the \macs0416 field, we can not
plausibly generate two events with similar decay timescales at
distinct locations on the sky.  This is because a caustic-crossing
transient event must necessarily appear at the location of the lensing
critical curve, but in this case the critical curve most likely passes
between the two observed \spock locations. At best, a caustic crossing
could account for only one of the \spock events, not both.
