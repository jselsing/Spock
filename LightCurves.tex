\section{Light Curves}\label{sec:LightCurves}

The two \spock\ events exhibited very similar light curves, as shown
in Figures \ref{fig:LightCurves}.  The January event, \spockone, rises
to peak luminosity and fades back to quiescence in only $\sim$7 days,
and the August event, \spocktwo, rises and falls in $<20$ days.
  
  
Due to the rapid decline timescale, no observations were collected for
either event that unambiguously show the declining portion of the
light curve. Therefore, we must make some assumption for the shape of
the light curve in order to quantify the peak luminosity and the
corresponding timescales for the rise and the decline.  We first
approach this with a simplistic model that is piece-wise linear in
magnitude vs time.  Figure~\ref{fig:LinearLightCurveFits} shows
examples of the resulting fits for the two events.  For each fit we
use only the data collected within 3 days of the brightest observed
magnitude, which allows us to fit a linear rise separately for the
F606W and F814W light curves for \spockone and the F125W and F160W
light curves for \spocktwo. To quantify the covariance between the
true peak brightness, the rise time and the decline timescale, we use
the following procedure:

\begin{enumerate}
\item make an assumption for the date of peak, $t_{\rm pk}$;
\item measure the peak magnitude at $t_{\rm pk}$ from the linear fit
  to the rising light curve data;
\item assume the source reaches a minimum brightness (maximum
  magnitude) of 30 AB mag at the epoch of first observation after the
  peak;
\item draw a line for the declining light curve between the assumed
  peak and the assumed minimum brightness;
\item use that declining light curve line to measure the timescale for
  the event to drop by 3 magnitudes, $t_3$;
\item make a new assumption for $t_{\rm pk}$ and repeat.
\end{enumerate}

As shown in Figure~\ref{fig:LinearLightCurveFits}, the resulting
piece-wise linear fits are simplistic, but nevertheless approximately
capture the observed behavior for both events.  Furthermore, since
this toy model is not physically motivated, it allows us to remain
agnostic for the time being as to the astrophysical source(s) driving
these transients.  From these fits we can see that \spockone most
likely reached a peak magnitude between 25 and 26.5 AB mag in both
F814W and F435W, and had a decline timescale $t_3$ of less than 2 days
in the rest-frame. The observations of \spocktwo provide less
stringent constraints, but we see that it had a peak magnitude between
23 and 26.5 AB mag in F160W and exhibited a decline time of less than
seven days.  These fits also illustrate the generic fact that a higher
peak brightness corresponds to a longer rise time and a faster decline
timescale, independent of the specific model used.  These results are
not substantially affected by changes to the arbitrary constraints we
placed on these linear fits, such as the maximum post-peak magnitude
or the number of pre-peak data points used for the rising light curve
fits.

With measurements of the key observational parameters of the \spock
light curves now in hand, we would like to compare these to the
characteristics of known astrophysical transients.  We must first,
however, introduce a correction for the lensing magnification, and
evaluate whether or how these sources might be spatially or temporally
related in the source plane.

