\section{Recurrent Nova Model}
\label{sec:RecurrentNovaModel}

In this section we argue that the \spock events were most likely
generated by two separate outbursts from a single stellar source, such
as a Recurrent Nova (RN) system. A nova explosion can occur in a
binary star system in which the more massive star is a white dwarf
that accretes matter from its companion, which may be a main sequence
dwarf or evolved giant star overfilling its Roche Lobe. The white
dwarf builds up a dense layer of H-rich material on its surface until
the high pressure and temperature triggers nuclear fusion, resulting
in a surface explosion that causes the white dwarf to brighten by
several orders of magnitude, but does not completely disrupt the
star. In a RN system, the mass transfer from the companion to the
white dwarf restarts after the explosion, so the cycle may begin again
and repeat after a period of months or years.  As we demonstrate
below, this model can accommodate all of the available evidence.
However, we will see that the luminosity and light curve of the \spock
events would imply that this is physically a very extreme RN system.

\subsection{Decline Rate}

A first line of evidence supporting the nova hypothesis comes from the
\spock light curves. Some RN light curves are similar in shape to the
\spock episodes, exhibiting a sharp rise ($<10$ days in the
rest-frame) and a similarly rapid decline.
Figure~\ref{fig:RecurrentNovaLightCurveComparison} compares the \spock
light curves to template light curves from RNe within our galaxy and
in M31.  There are 10 known RNe in the Milky Way galaxy, and 7 of
these exhibit outbursts that decline rapidly, fading by 2 magnitudes
in less than 10 days \citep{Schaefer:2010}
% U Sco, V2487 Oph, V394 CrA, T CrB, RS Oph, V745  Sco, and V3890 Sgr.
The gray shaded region in
Figure~\ref{fig:RecurrentNovaLightCurveComparison} encompasses the V
band light curve templates for all 7 of these events, from
\citet{Schaefer:2010}.  The Andromeda galaxy (M31) also hosts at least
one RN with a rapidly declining light curve.  The 2014 eruption of
this well-studied nova, M31N 2008a-12, is shown as a solid black line
in Figure~\ref{fig:RecurrentNovaLightCurveComparison}, fading by 2
mags in less than 3 days.  This comparison demonstrates that the
sudden disappearance of both of the \spock transient events is fully
consistent with the eruptions of known RNe in the local universe.

\subsection{Recurrence}

The second reason to consider the RN model is that it provides a
natural explanation for having two separate explosions that are
coincident in space but not in time.  If \spock is a RN, then the two
observed episodes can be attributed to two distinct nova eruptions,
and the gravitational lensing time delay does not need to match the
observed 8 month separation between the January and August 2014
appearances.  Alternatively, one might suppose that the two \spock
events are actually two images of the same physical episode, appearing
to us separately only because of the lensing delay -- as was the case
for the 5 images of SN Refsdal \citep{Kelly:2015a,Kelly:2016}.
% For SN Refsdal
% the lens models were collectively very accurate in predicting the time
% delays between the 4 images in the Einstein cross configuration
% \citep{Treu:2015b,Rodney:2016} and the return as a fifth image
% \citep{Kelly:2016}. The lens modeling for \spock uses much of the same
% methodology, so there is no a priori reason to be suspicious of the
% time delay predictions.

However, we have seen in Section~\ref{sec:LensingModels} that none of
the \macs0416 lens models predict an 8 month time delay between
appearances in image 11.1 and 11.2.  This is represented in
Figure~\ref{fig:SpockDelayPredictions}, where we have plotted the
light curves for the two transient events, along with shaded vertical
bars marking the time delay predictions of all models.
%The lens models are broadly consistent with each other, predicting
%that the lensing time delay between images 11.1 and 11.2 is on the
%order of $\pm$60 days, far short of the 238 day lag that was observed
%between \spockone\ and \spocktwo.
To accept the alternative single-explosion explanation
for \spock, we would have to assume that a large systematic bias is
similarly affecting all of the lens models.  While we cannot rule out
such a bias, the consistency of the lens modeling makes a recurrent
explosion model more tenable.

Although {\it qualitatively} consistent, the RN model is strained by a
quantitative assessment of the recurrence period. If \spock is indeed
a RN at $z=1$, then the recurrence timescale in the rest-frame is
$120\pm30$ days ($3-5$ months), where the uncertainty accounts for the
$1\sigma$ range of modeled gravitational lensing time delays.  This
would be a singularly rapid recurrence period, significantly faster
than all 11 RNe in our own galaxy, which have recurrence timescales
ranging from 15 years (RS Oph) to 80 years (T CrB). For the 5 galactic
RNe with a rapidly declining outburst light curve (U Sco, V2487 Oph,
V394 CrA, T CrB, and V745 Sco), the median recurrence timescale is 21
years.  The fastest measured recurrence timescale belongs to the
Andromeda galaxy nova M31N 2008a-12, which exhibits a new outburst
every year. Although this M31 record-holder demonstrates that very
rapid recurrence is possible, classifying \spock as a RN would still
require a very extreme mass transfer rate to accommodate the $<1$ year
recurrence.

\subsection{Luminosity}

To infer a peak luminosity for both \spock events, we combine the
linear fits to the light curves (shown in
Figure~\ref{fig:LightCurveLinearFits}) with the predicted range of
lensing magnifications (Figure~\ref{fig:LensModelContours}. Given any
assumption for the time of peak brightness, we use the extrapolated
light curve fits to deduce the peak magnitude in an observed bandpass
and a corresponding rise-time and decline-time measurement.  We then
convert this observed peak magnitude to a luminosity (e.g., in
erg/s) by correcting for the luminosity distance (assuming a standard
\LCDM cosmology) and the lensing magnification.

Figure~\ref{fig:PeakLuminosityDeclineTime} shows the resulting
constraints on the peak luminosity and the decline time, which we
quantify as $t_3$, the time over which the transient declines by 3
magnitudes.  Shaded green and red bands for the \spock events
encompass the allowed $1\sigma$ range for the lensing magnification as
reported in Table~\ref{tab:LensModelPredictions}, covering $\mu\sim10$
to $\mu\sim100$.  The two events are largely consistent with each
other, and if both events are representative of a single RN system
then the most likely peak luminosity and decline time (the region with
the most overlap) would be $L_{\rm pk}\sim10^{41}$ ergs/s and
$t_3\sim1.8$ days.  This is substantially faster and brighter than any
known nova system in the Milky Way or M31. 

The seminal work of \citet{Zwicky:1936} and \citet{McLaughlin:1939}
first showed that more luminous novae within the Milky Way tend to
have more rapidly declining light curves, which is now the basis of
the maximum-magnitude versus rate-of-decline (MMRD) relationship. The
basic form of the MMRD relation has been theoretically attributed to a
dependence of the peak luminosity on the mass of the accreting white
dwarf \citep[e.g.][]{Livio:1992}.  Studies of extragalactic novae
reaching as far as the Virgo cluster have shown that the MMRD relation
is broadly applicable to all nova populations, though with significant
scatter
\citep[e.g.][]{Ciardullo:1990,DellaValle:1995,Ferrarese:2003,Shafter:2011}.
Amidst that scatter, there may also be sub-populations of novae that
deviate from the traditional MMRD form \citep{Kasliwal:2011}.

As Figure~\ref{fig:PeakLuminosityDeclineTime} shows, the \spock events
are extreme outliers relative to the MMRD relation, regardless of the
assumed values for magnification and time of peak brightness. The most luminous novae observed in the local universe were all somewhat slower in their decline rate 

SN 2010U has t2 = 3.5 ± 0.3 days and the rise time is unconstrained.

t2 = 6 ± 1 days (L91; Schwarz et al. 2001) and t2 = 9.5 days (M31N; Shafter et al. 2009). By comparison, 

L91 (Della Valle 1991; Schwarz et al. 2001; Williams et al. 1994) and M31N-2007-11d (Shafter et al. 2009)

The preferred peak luminosity of
$10^{41}$ erg s$^{-1}$ that we have inferred for \spock would imply
that this is among the most luminous novae ever observed.



extremely luminous novae \citep{Czekala:2013}
was an Fe II nova, inconsistent with the usual picture of He/N novae as the brightest with the most massive WDs.

Two other luminous Fe ii type novae have been studied extensively: , hereafter M31N.

The rise to maximum of L91 is among the longest for novae on record, with a peak of Mv = −10.0 mag. The light curve of L91 shown here is drawn from the photometry published in the circulars (Shore et al. 1991; Gilmore 1991; Gilmore et al. 1991; Liller et al. 1991; Della Valle et al. 1991). Shafter et al. (2009) set a lower limit of four days on the rise time for M31N from quiescence to a maximum light of MV ≃ −9.5 mag.

Both novae declined rapidly from maximum light with 



Furthermore, the spectroscopic classification
of Novae is also correlated with their luminosity and light curve
decline time: those showing prominent He/N features are brighter and
fade faster than those with spectra dominated by Fe II lines.

fast and faint novae don't follow the MMRD \citep{Kasliwal:2011}


   These observations are consistent with the classification of 

consistent with the
short separation between observations of \spock. 


Shafter et al 2011:

``more luminous novae generally fade the fastest and [...]  He/N novae
are typically faster and brighter than their Fe II counterparts. In
addition, we find a weak dependence of nova speed class on position in
M31, with the spatial distribution of the fastest novae being slightly
more extended than that of slower novae.''


Recurrent novae make up roughly 25\% of the nova population
(masquerading as CNe \citep{Pagnotta:2014}.
