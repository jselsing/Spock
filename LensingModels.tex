\section{Lensing Models}\label{sec:LensingModels}

To interpret the observed light curves and the timing of these two
events, we use \textcolor{red}{four} cluster mass models to provide
estimates of the magnification and time delays from gravitational
lensing.  We identify each by the last name of the principal
investigator of the development team:
\textcolor{red}{NOTE: THESE DESCRIPTIONS ARE PROBABLY INCORRECT.  MODELERS, PLEASE FIX AS NEEDED.}

\begin{itemize}
\item{\it Diego} Created with the {\tt WSLAP+} software
  \citep{Sendra:2014}: Weak and Strong Lensing Analysis Package plus
  member galaxies (Note: no weak-lensing constraints used for this
  MACS J0416 model). Interactive online model exploration available at
  \url{http://www.ifca.unican.es/users/jdiego/LensExplorer}.
\item{{\it Jauzac} Generated with the {\tt LENSTOOL} software
  \citep{Jullo:2007},\footnote{\url{http://projects.lam.fr/repos/lenstool/wiki}}}
  using strong- and weak-lensing constraints.  This model makes a
  light-traces-mass assumption and parameterizes cluster components
  using Navarro-Frenk-White (NFW) density profiles
  \citep{Navarro:1997}.
\item{{\it Williams} An adaptive grid model developed using the {\tt
    GRALE} software tool
  \citep{Liesenborgs:2006,Liesenborgs:2007,Mohammed:2014}, which uses
  a genetic algorithm to reconstruct the cluster mass profile with an
  arrangement of projected Plummer \citeyear{Plummer:1911} density
  profiles.}
\item{{\it Zitrin} A model with strong- and weak-lensing constraints, built using the PIEMD+eNFW parameterization for density profiles as in \citet{Zitrin:2009a}.
\end{itemize}
    

The {\it Williams} and {\it Zitrin} models were originally distributed
as part of the Hubble Frontier Fields lens modeling
project,\footnote{For more details, see
  \url{https://archive.stsci.edu/prepds/frontier/lensmodels/}} in
which models were generated based on data available before the start
of the HFF observations to enable rapid early investigations of lensed
sources. The {\it Jauzac} model is an updated version of the model
developed for that HFF modeling effort by the CATS team.  In all cases
the lens modelers made use of strong-lensing constraints
(multiply-imaged systems and arcs) derived from HST imaging collected
as part of the CLASH program (PI:Postman, HST PID:12459,
\citealt{Postman:2012}). These models also made use of spectroscopic
redshifts in the cluster field from \citet{Mann:2012},
\citet{Christensen:2012}, and \citet{Grillo:2015}.  Input weak-lensing
constraints also made use of data collected at the Subaru Telescope by
PI K. Umetsu (in prep) and archival imaging.
