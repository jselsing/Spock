\section{Constraints from Lens Model}\label{sec:LensingModels}

To interpret the observed light curves and the timing of these two
events, we use \textcolor{red}{four} cluster mass models to provide
estimates of the magnification and time delays from gravitational
lensing.  We identify each by the last name of the principal
investigator of the development team:
\textcolor{red}{NOTE: THESE DESCRIPTIONS ARE PROBABLY INCORRECT.  MODELERS, PLEASE FIX AS NEEDED.}

\begin{itemize}
\item{\it Diego:} Created with the {\tt WSLAP+} software
  \citep{Sendra:2014}: Weak and Strong Lensing Analysis Package plus
  member galaxies (Note: no weak-lensing constraints used for this
  MACS J0416 model). Interactive online model exploration available at
  \url{http://www.ifca.unican.es/users/jdiego/LensExplorer}.
\item{{\it Jauzac:} The model of \citet{Jauzac:2014}, generated with the {\tt LENSTOOL} software
  \citep{Jullo:2007},\footnote{\url{http://projects.lam.fr/repos/lenstool/wiki}}}
  using strong- and weak-lensing constraints.  This model makes a
  light-traces-mass assumption and parameterizes cluster components
  using Navarro-Frenk-White (NFW) density profiles
  \citep{Navarro:1997}.
\item{{\it Williams:} An adaptive grid model developed using the {\tt
    GRALE} software tool
  \citep{Liesenborgs:2006,Liesenborgs:2007,Mohammed:2014}, which uses
  a genetic algorithm to reconstruct the cluster mass profile with an
  arrangement of projected Plummer \citeyear{Plummer:1911} density
  profiles.}
\item{{\it Zitrin:} A model with strong- and weak-lensing constraints, built using the PIEMD+eNFW parameterization for density profiles as in \citet{Zitrin:2009a}.
\end{itemize}
    
\smallskip
The {\it Williams} and {\it Zitrin} models were originally distributed
as part of the Hubble Frontier Fields lens modeling
project,\footnote{For more details, see
  \url{https://archive.stsci.edu/prepds/frontier/lensmodels/}} in
which models were generated based on data available before the start
of the HFF observations to enable rapid early investigations of lensed
sources. The {\it Jauzac} model is an updated version of the model
developed for that HFF modeling effort by the CATS team.  In all cases
the lens modelers made use of strong-lensing constraints
(multiply-imaged systems and arcs) derived from HST imaging collected
as part of the CLASH program (PI:Postman, HST PID:12459,
\citealt{Postman:2012}). These models also made use of spectroscopic
redshifts in the cluster field from \citet{Mann:2012},
\citet{Christensen:2012}, and \citet{Grillo:2015}.  Input weak-lensing
constraints also made use of data collected at the Subaru Telescope by
PI K. Umetsu (in prep) and archival imaging.

These models provide estimates for the absolute magnifications due to gravitational lensing from the MACSJ0416 cluster, reported in Table~\ref{tab:magnifications}.  Although the models generally agree on magnification values in the range 10-60 for both events, the uncertainties on these magnifications are very large, due primarily to the close proximity of the \spock\ events to the lensing critical curve for sources at $z=1$ \textcolor{red}{A GOOD CITATION FOR THIS GENERAL TRAIT OF MODEL UNCERTAINTY INCREASING NEAR CRITICAL CURVES?}.

From each model we also extract two time delay predictions.  First we consider the \spockone\ event, which appeared in January 2014 in host image 11.2.  That same transient episode must have also appeared in the adjacent host galaxy image 11.1, but at a different time, due primarily to the \citet{Shapiro:1964} delay.  We ask when that event should have been observed in the adjacent host image 11.1, and label this as the "\spockone-11.1" delay.  The second prediction extracted from each model is the "\spocktwo-11.2 delay": the time difference between the August 2014 \spocktwo\ event and its expected appearance in host image 11.2.   If the two events are actually two images of the same physical episode, appearing separately only because of gravitational lensing, then both of these time delays should be consistent with the observed 8 month separation between the January and August 2014 appearances.   

In fact, we find that none of the lensing models predict an 8 month time delay from image 11.1 to 11.2.  This is represented in Figure~\ref{fig:SpockDelayPredictions}, where we have plotted the two transient events, along with shaded bars demarcating the time delay predictions of all models.   The models are broadly consistent with each other, predicting that the lensing time delay from image 11.1 to 11.2 is on the order of $\pm$50 days, far short of the 238 day lag that we observed between \spockone\ and \spocktwo.  From this we conclude that the two observed events are not gravitational echoes of a single explosive transient episode, but instead must have originated as two distinct physical events in the source plane.  We are left then with two possible scenarios: (a) the two events are not physically associated, in which case they may each be the result of a separate catastrophic explosion like a supernova or kilonova that would leave the progenitor system completely disrupted; or (b) the two events originated from the same astrophysical system, which must therefore be a source of recurrent explosive transient episodes.   We evaluate physical systems that could match these two scenarios in the following sections.

