\section{Constraints from Lens Model}\label{sec:LensingModels}

To interpret the observed light curves and the timing of these two
events, we use \textcolor{red}{five} cluster mass models to provide
estimates of the magnification and time delays from gravitational
lensing.  We identify each by the last name of the principal
investigator of the development team:
\textcolor{red}{NOTE: THESE DESCRIPTIONS ARE PROBABLY INCORRECT.
  MODELERS, PLEASE FIX AS NEEDED.}


\bigskip
\begin{itemize}
\item{\it Diego:} Created with the {\tt WSLAP+} software
  \citep{Sendra:2014}: Weak and Strong Lensing Analysis Package plus
  member galaxies (Note: no weak-lensing constraints used for this
  MACS J0416 model). Interactive online model exploration available at
  \url{http://www.ifca.unican.es/users/jdiego/LensExplorer}.
\item{{\it Jauzac:} The model of \citet{Jauzac:2014}, generated with
  the {\tt LENSTOOL} software
  \citep{Jullo:2007},\footnote{\url{http://projects.lam.fr/repos/lenstool/wiki}}}
  using strong- and weak-lensing constraints.  This model makes a
  light-traces-mass assumption and parameterizes cluster components
  using Navarro-Frenk-White (NFW) density profiles
  \citep{Navarro:1997}.
\item{\it Oguri:} The model of \citet{Kawamata:2015}, built using the
  {\tt GLAFIC}
  software\footnote{\url{http://www.slac.stanford.edu/~oguri/glafic/}}
  with strong-lensing constraints.
\item{{\it Williams:} An adaptive grid model developed using the {\tt
    GRALE} software tool
  \citep{Liesenborgs:2006,Liesenborgs:2007,Mohammed:2014}, which uses
  a genetic algorithm to reconstruct the cluster mass profile with an
  arrangement of projected Plummer \citeyear{Plummer:1911} density
  profiles.}
\item{{\it Zitrin:} A model with strong- and weak-lensing constraints,
  built using the PIEMD+eNFW parameterization for density profiles as
  in \citet{Zitrin:2009a}.}
\end{itemize}
\bigskip    


The {\it Williams} and {\it Zitrin} models were originally distributed
as part of the Hubble Frontier Fields lens modeling
project,\footnote{For more details, see
  \url{https://archive.stsci.edu/prepds/frontier/lensmodels/}} in
which models were generated based on data available before the start
of the HFF observations to enable rapid early investigations of lensed
sources. The {\it Jauzac} model is an updated version of the model
developed for that HFF modeling effort by the CATS team.  In all cases
the lens modelers made use of strong-lensing constraints
(multiply-imaged systems and arcs) derived from HST imaging collected
as part of the CLASH program (PI:Postman, HST PID:12459,
\citealt{Postman:2012}). These models also made use of spectroscopic
redshifts in the cluster field from \citet{Mann:2012},
\citet{Christensen:2012}, and \citet{Grillo:2015a}.  Input
weak-lensing constraints also made use of data collected at the Subaru
Telescope by PI K. Umetsu (in prep) and archival imaging.
\citet{Priewe:2016} provides a more complete description of the
methodology of each model and a comparison of the magnification
predictions and uncertainties across the entire \macs0416 field.

These models provide estimates for the absolute magnifications due to
gravitational lensing from the \macs0416\ cluster, reported in
Table~\ref{tab:LensModelPredictions}.  Collectively, the models
predict magnification values between about $\mu=10$ and $\mu=100$ for
both events. This wide range is due primarily to the close proximity
of the \spock\ events to the lensing critical curve for sources at
$z=1$.  Note that the magnifications for \spockone\ and \spocktwo\ are
highly correlated.  A variation of a given lens model that moves the
critical curve closer to the position of \spockone\ would drive the
magnification of that event much higher (toward $\mu_1\sim100$), but
that would also have the effect of moving the critical curve farther
from \spocktwo\, which would necessarily drive its magnification
downward (toward $\mu_2\sim10$).  \todo{add a citation for the general
  trait of model uncertainty increasing near critical curves}

%\renewcommand{\arraystretch}{1.5}
\begin{deluxetable}{lccccc}\label{tab:LensModelPredictions}
\tablewidth{\linewidth} \tablecolumns{6} \tablecaption{Lens model
  predictions for time delays and
  magnifications\label{tab:LensModelPredictions}} \tablehead{
  \colhead{Model} & \colhead{$\Delta t_{\rm NW:SE}$} & \colhead{$\Delta
    t_{\rm NW:11.3}$} & \colhead{$\mu_1$} & \colhead{$\mu_2$} &
  \colhead{$\mu_3$}\\ \colhead{} & \colhead{(days)} &
  \colhead{(years)} & \colhead{} & \colhead{} & \colhead{} }
\startdata
Diego & -48$\pm$10 & 0.8 & 35$\pm$20 & 30$\pm$20 &\\[0.5em]
Jauzac & -16 $\pm13$ & -2.4 $^{+0.5}_{-0.6}$ & 37 $\pm3$ & 18 $\pm2$ & 4.6 $\pm0.1$\\[0.5em]
Oguri & 4.1 $^{+5.5}_{-3.4}$ & -5.0 $^{+0.5}_{-0.6}$ & 29 $^{+43}_{-10}$ & 84 $^{+103}_{-38}$ & 3.0 $^{+0.2}_{-0.2}$\\[0.5em]
Williams & -10 $^{+1}_{-7}$ & -2.5 $^{+1.0}_{-3.1}$ & 13 $^{+11}_{-6}$ & 12 $^{+9}_{-5}$ & 3.1 $^{+2.2}_{-0.9}$\\[0.5em]
Zitrin & 42 $^{+13}_{-9}$ & -3.7 $\pm0.3$ & 90 $^{+61}_{-27}$ & 32 $^{+8}_{-10}$ & 3.6 $^{+0.2}_{-0.5}$\\
\enddata
\tablecomments{Time delays give the predicted delay relative to an
  appearance in the NW host image, 11.2. Positive (negative) values indicate the
  NW image is the leading (trailing) image of the pair.}
\end{deluxetable}
%\renewcommand{\arraystretch}{1.}

From each model we also extract two time delay predictions, given in
Table~\ref{tab:LensModelPredictions}.  We report all time delays
relative to the \spockone\ event, which appeared in January 2014 in
host image 11.2.  That same transient episode would have appeared at
different times in host galaxy images 11.1 and 11.3, due primarily to
the \citet{Shapiro:1964} delay.  The $\Delta t_{\rm NW:SE}$ column in
Table~\ref{tab:LensModelPredictions} gives the model predictions for
the number of days between the appearance of the \spockone\ event in
the NW host image (11.2) and the date when it should have been
observable in the adjacent SE host image (11.1).  The opposite value
would give the time difference between the August 2014
\spocktwo\ event and its expected appearance in host image 11.2.
Table~\ref{tab:LensModelPredictions} also reports the predicted time
delay (in years) between appearance in the NW host image 11.2 and the
more widely separated image 11.3.

Figure~\ref{fig:LensModelContours} presents probability distributions
derived from these models for the three magnifications and two time
delay values of interest.  These distributions were derived by
combining the Monte Carlo chains from the Jauzac, Oguri, Williams and
Zitrin models, with weighting applied to account for the different
number of model iterations in each chain. Four of the five models
agree that host image 11.3 is the leading image, appearing some 2--6
years before the other two images.  The models do not agree on the
arrival sequence of images 11.1 and 11.2: some have the NW image 11.2
as a leading image, and others have it as a trailing image.  However,
the models do consistently predict that the separation in time between
those two images should be roughly in the range of 1 to 60 days.
