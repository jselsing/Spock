\section{Observations}\label{sec:Observations}

The transient \spock\ was discovered in HST imaging collected as part
of the Hubble Frontier Fields (HFF) survey (HST-PID:13496, PI:Lotz), a
multi-cycle program observing 6 massive galaxy clusters and associated
``blank sky'' parallel fields.  Several HST observing programs have
provided additional observations supplementing the core HFF program.
One of these is the FrontierSN program (HST-PID:13386, PI:Rodney),
which aims to identify and study explosive transients found in the HFF
and related programs.  The FrontierSN team discovered \spock\ in two
separate HFF observing campaigns on the galaxy cluster
\MACS0416\ (hereafter, MACS0416).  The first was an imaging campaign
in January, 2014 during which the MACS0416 cluster field was observed
in optical bands using the Advanced Camera for Surveys Wide Field
Camera (ACS-WFC).  The second concluded in August, 2014, and imaged
the cluster with the infrared detector of HST's Wide Field Camera 3
(WFC3-IR).

To discover transient sources, the FrontierSN team processes each new
epoch of HST data through a difference imaging
pipeline\footnote{\url{https://github.com/srodney/sndrizpipe}}, using
archival HST images to provide reference images (templates) which are
subtracted from the astrometrically registered HFF images. In the case
of MACS0416, the templates were constructed from images collected as
part of the Cluster Lensing And Supernova survey with Hubble (CLASH,
HST-PID:12459, PI:Postman). The resulting difference images are
visually inspected for new point sources, and any new transients of
interest (primarily supernovae, SNe) are followed up with additional
HST imaging or ground-based spectroscopic observations as needed.  For
a more complete description of the operations of the FrontierSN
program, see \citep{Rodney:2015a}.

The follow-up observations for \spock\ included HST observations in
infrared and optical bands using the WFC3-IR and ACS-WFC detectors,
respectively, as well as spectroscopy of the \spock\ host galaxy using
the Very Large Telescope (VLT) and the Visible Multi-object
Spectrograph (VIMOS, \textcolor{red}{citation?}), and also using the
X-Shooter cross-dispersed echelle spectrograph
\citep{Vernet:2011}. The X-Shooter observations
(\textcolor{red}{Program ID?}, PI:Hjorth) were taken on October 19th,
21st and 23rd (2014) for a total of 4.5 hours, with the slit centered
on the position of \spock2.  The spectrum did not provide any
detection of the transient source itself (as we will see below, it had
already faded back to its quiescent state by that time).  However, it
did provide an unambiguous redshift for the host galaxy of
$z=1.0054\pm0.0002$ from \Ha\ and \ion{O}{[ii]}.  These line
identifications are consistent with two measures of the photometric
redshift of the host: $z=1.00+-0.02$ from the BPZ algorithm
\citep{Benitez:2000}, and $z=0.92\pm0.05$, derived using the EAZY
program \citep{Brammer:2008}.


TODO: add a table with photometry.


  
  

  
  
  
  