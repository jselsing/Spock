\section{Host Galaxy}
\label{sec:HostGalaxy}

Before we can examine astrophysical models that might explain these
transients, we first must examine whether the two transients
originated from the same physical location in the source plane. One
way to test this is to compare the properties of the \spock host
galaxy at the location of each event.  To that end, we used the
technique of ``pixel-by-pixel'' SED fitting as described in
\citet{Hemmati:2014} to determine rest-frame colors and stellar
properties in a single resolution element centered at each transient
location.  For this purpose we used the deepest possible stacks of HST
images, comprising all available data except those images where the
transient events were present.  The resulting maps of stellar
population properties are shown in Figure~\ref{fig:HostProperties}.
Table~\ref{tab:HostProperties} reports measurements of the three
derived stellar population properties (color, mass, age) from host
images 11.1, 11.2 and 11.3.  In 11.1 and 11.2 these measurements were
extracted from the central pixel at the location of each of the two
\spock events.  Assuming the lensing magnification here ranges from
$\mu=10$ to 100 (see Section~\ref{sec:LensingModels}, this corresponds
to a size on the source plane between 6 and 600 pc$^2$.  For host
image 11.3 we report the stellar population properties dervide from
the pixel at the center of the galaxy.  With a magnification of
$\sim$3 to 5, the extraction region covers roughly 2000 to 6000
pc$^2$.


\begin{deluxetable}{lccc}
  \tablewidth{\linewidth}
  \tablecolumns{6}
  \tablecaption{Properties of the local stellar population in the \spock host galaxy, from SED fitting.}
  \tablehead{ {Host image:} & \colhead{11.1} & \colhead{11.2} & \colhead{11.3}\\
{Location:}   & \colhead{\spocktwo} & \colhead{\spockone} & \colhead{center}}
\startdata
$(U-V)_{\rm rest}$            & 0.69$^{+0.2}_{-0.05}$  & 0.52$^{+0.15}_{-0.10}$      & 0.39$\pm$0.05  \\
$\log[\Sigma (M_*/\Msun)]$  & 7.14 $\pm$ 0.15   & 7.14 $\pm$ 0.15     & 7.04 $\pm$ 0.10   \\
Age (Gyr)                   & 0.292$\pm$0.5 &   0.290$\pm$0.5 &  0.292$\pm$0.5  
\enddata
\label{tab:HostProperties}
\end{deluxetable}

The reported uncertainties for these derived stellar properties in
Table~\ref{tab:HostProperties} reflect only the measurement errors
from the SED fitting, and do not attempt to quantify potential
systematic biases.  Such biases could arise, for example, from color
differences in the background light, which is dominated by the cluster
galaxies and varies significantly across the \macs0416 field.  Such a
bias might shift the absolute values of the parameter scales for any
given host image (e.g., making the galaxy as a whole appear bluer,
more massive and younger). However, the gradients across any single
host image are unlikely to be driven primarily by such systematics.

Figure~\ref{fig:HostProperties} and Table~\ref{tab:HostProperties}
show that the measured values of the color, stellar mass, and age at
the two \spock locations are mutually consistent. Thus, it is
plausible to assume that the two positions map back to the same
physical location at the source plane.  Comparing those two locations
to the center of the galaxy as defined in image 11.3, we see only a
mild tension in the rest-frame U-V color. This comparison therefore
can not quantitatively rule out the possibility that the two transient
events are located at the center of the galaxy. However, the maps
shown in Figure~\ref{fig:HostProperties} do show a gradient in both
U-V color and stellar age. For both images 11.1 and 11.2 the bluest
and youngest stars (U-V$\sim$0.3, $\tau\sim$280 Myr) are localized in
knots near the extreme ends of each image, well separated from either
of the \spock transient events.  In the less distorted host image 11.3
the bluer and younger stars are concentrated near the center. Taken
together, these color and age gradients suggest that the two
transients are not coincident with the center of their host galaxy.

In addition to the HST imaging data, we also have spatially resolved
spectroscopy from the MUSE integral field data. The only significant
spectral line feature for the \spock host is the \ionline{O}{[ii]}
($\lambda\lambda$ 3727, 3729) doublet, observed at 7474 and 7478
\AA. Figure~\ref{fig:MUSEOIISequence} shows the observed
\ionline{O}{[ii]} lines at 10 positions along the length of the arc,
which comprises images 11.1 and 11.2.  At each position the lines were
extracted using apertures with a radius of 0.6\arcsec, so adjacent
extractions are not independent, but extractions at the center of
11.1 and the center of 11.2 have no overlap. Each extraction has been
normalized to show a peak line flux at unity, so that the line
profiles and the doublet line ratios may be more easily
compared. \todo{Make a table of line ratios and say something about
  whether spock1 and spock2 are consistent with being coincident at
  the source plane}


\begin{deluxetable}{lccc}
	\tablewidth{\linewidth}
	\tablecolumns{6}
	\tablecaption{Line-ratios of \ionline{O}{[ii]} at positions across the host.}
	\tablehead{ {Host image:} & \colhead{11.1} & \colhead{11.2} & \colhead{11.3}\\
		{Location:}   & \colhead{\spocktwo} & \colhead{\spockone} & \colhead{center}}
	\startdata
	\ionline{O}{[ii]} $\lambda$3726 (XSHOOTER) $^{a}$          & 0.30$^{+0.04}_{-0.04}$  & n/a  & n/a  \\
	\ionline{O}{[ii]}$\lambda$3729 (XSHOOTER) $^{a}$   & 0.49$^{+0.04}_{-0.04}$   & n/a & n/a   \\
	\ionline{O}{[ii]}$_{ratio}$   (XSHOOTER)          & 1.6$^{+0.2}_{-0.2}$ & n/a & n/a
	\enddata
	
	\tablecomments{ $^{a}$ Measured flux is in units of 10$^{-17}$ erg s$^{-1}$ cm$^{-2}$. 
	}
	
	\label{tab:HostProperties}
\end{deluxetable}